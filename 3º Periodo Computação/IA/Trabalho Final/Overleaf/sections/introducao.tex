\section{Introdução}
\label{sec:introducao}

\subsection{Apresentação}
A análise de dados apresentadas a seguir foram feitas com base nos conhecimentos aprendidos no decorrer da disciplina de Inteligência Artificial. Foram utilizados como ferramentas o excel para registrar os valores obtidos, KNIME Analytics Plataform para fazer as análises de todas as bases e a ferramenta OVERLEAF para a criação deste relátorio.

\subsection{Metodologia}
Foram feitos 3 testes com 100 interações em cada base e outros 3 testes com 400 interações em cada base. File Reader, Category to Number, Missing Value, Normalizer e Partitioning são os nodes presentes na execução das análises.

\subsection{Procedimentos}
O processo para definir qual a melhor base foi a seguinte:
\begin{itemize}
    \item Cada um dos testes foram realizados com um parâmetro pré-definido;
    \item Com base nos resultados tiramos a média entre os 6 testes feitos;
    \item Os melhores são guardados para a execução de novos testes;
    \item De acordo com os resultados obtidos, define-se qual é o melhor.
\end{itemize}