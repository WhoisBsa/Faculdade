\section{Qual o conceito de data mining ou mineração de dados?}

Processo de explorar grandes quantidades de dados à procura de padrões consistentes.
\cite{CetaxDeal}

\section{Cite e descreva duas aplicações práticas de data mining.}

\noindent{\textbf{Educação}}

Inspiradas pelo programa mundial OLPC (One Laptop per Child) proposto pelo Laboratório de Mídias do MIT, as iniciativas UCA (Projeto e Programa "Um Computador por Aluno") vêm promovendo a introdução de laptops de baixo custo em centenas de escolas brasileiras. Em ambas as iniciativas, gestores e docentes brasileiros carecem de informações sobre a efetiva contribuição do uso de computadores nas escolas. Diante deste cenário, participamos do desenvolvimento de um sistema de informação chamado MEMORE para tentar suprir tal carência. Este sistema permite a coleta e a mineração de dados sistemática (tarefa de Descoberta de Associações) de informações sobre como os laptops UCA têm sido pedagogicamente utilizados e os efeitos decorrentes dessa utilização no aprendizado dos alunos. Participante das iniciativas UCA desde o início de sua aplicação, a Secretaria Municipal de Educação de Piraí, no estado do Rio de Janeiro, vem utilizando, com sucesso, o MEMORE em caráter experimental em suas escolas.  
\newline

\noindent{\textbf{Energia}}

Um projeto importante na área energética (sobretudo em épocas de poucas chuvas) teve como objetivo aplicar técnicas de Data Mining para geração de modelos que façam a previsão de demanda de consumo de energia elétrica por regiões (tarefa de Regressão/Previsão de Séries Temporais). Para tanto, foram utilizados registros de consumo de energia elétrica ao longo de períodos anteriores.
\cite{ComputerWorld}

\section{Qual a diferença entre dado, informação e conhecimento?}

\noindent \textbf{Data}

Dados são quaisquer fatos, números ou textos que podem ser processados por um computador. Hoje, as organizações estão acumulando vastas e
quantidades crescentes de dados em diferentes formatos e diferentes
bases de dados. Estes incluem: 
\begin{itemize}
    \item dados operacionais ou transacionais, como vendas, custo,
estoque, folha de pagamento e contabilidade
    \item dados não operacionais, como vendas do setor, previsão
dados e dados macroeconómicos
    \item metadados - dados sobre os dados em si, como a lógica
design de banco de dados ou definições de dicionário de dados
\end{itemize}

\noindent \textbf{Informação}

Os padrões, associações ou relacionamentos entre todos esses dados
pode fornecer informações. Por exemplo, a análise do ponto de venda
Os dados de transação de venda podem gerar informações sobre quais produtos são vendendo e quando.

\noindent \textbf{Conhecimento}

A informação pode ser convertida em conhecimento sobre padrões e tendências futuras. 
Por exemplo, informações resumidas nas vendas de supermercados de varejo pode ser analisada à luz do esforços promocionais para fornecer conhecimento da compra do consumidor comportamento. 
Assim, um fabricante ou varejista poderia determinar quais itens são mais suscetíveis a esforços promocionais. 
\cite{ReviewlArticle}

\section{O que é KDD - Descoberta de Conhecimento em Bancos de Dados?}

Descoberta de conhecimento em bancos de dados (KDD) é o processo de descoberta de conhecimento útil a partir de uma coleção de dados. Essa técnica de mineração de dados amplamente usada é um processo que inclui preparação e seleção de dados, limpeza de dados, incorporação de conhecimento prévio em conjuntos de dados e interpretação de soluções precisas a partir dos resultados observados.
\cite{Techopedia}

\section{Explique e exemplifique os dois tipos de conhecimento estudados (explícito e tácito)}

Conhecimento explícito é o conhecimento que possuímos e de que temos consciência. É o conhecimento que somos capazes de documentar, é o conhecimento que as organizações conseguem armazenar.

O conhecimento tácito, ao contrário, é conhecimento que temos mas do qual não nos apercebemos. É conhecimento que adquirimos através da prática, da experiência, dos erros e dos sucessos. É conhecimento que não somos capazes de descrever nem documentar. É conhecimento que as organizações não podem utilizar depois fora das horas de trabalho dos seus colaboradores.
\cite{Kmol}

\section{Qual a importância da fase de preparação dos dados? Quais os principais problemas envolvendo coleta de dados do mundo real?}

Uma preparação de dados adequada torna possível extrair informações, auxiliando nas tomadas de decisões e nas soluções de problemas; Qualidade das informações disponíveis no mundo real.
\cite{IgtiBlog}

\section{Cite 1 exemplo de uma coleção de atributos e 3 objetos dessa coleção.}
Dados Nominais: dados nominais estão em formato alfabético e não em um inteiro.\cite{T4}

\noindent
\begin{table}[h]
    \begin{tabular}{c|c}
    \hline
         Dados Categóricos &  Docente, Professor Assistente, Professor \\\hline
         Estado & novo, pendente, trabalhando, completo, acabamento \\\hline
         Cores & preto, marrom, branco, vermelho \\
    \hline
    \end{tabular}
    \caption{Dados Nominais}
    \label{tab:my_label}
\end{table}


\section{Conceitue o processo de enriquecimento dos dados e cite um exemplo diferente dos que constam no material.}

É o processo de inserir dados de fontes externas ao conjunto de dados.

Imagine pode ter em sua base, dados demográficos, sócio-econômicos, sociais e comportamentais de todos seus clientes. Saber o poder de consumo, seus gostos, hábitos e afinidades nas redes sociais, seu comportamento e interação no seu site e qual a afinidade de cada cliente com seu produto. Isso é enriquecimento e inteligência de dados.
\cite{Livia}

\section{Conceitue o processo de melhoramento dos dados e cite um exemplo diferente dos que constam no material}

É o processo de realçar características dos dados sem adição de fontes externas

Conexão de um roteador wi-fi apenas pelo modelo não é suficiente para saber se é boa o suficiente. Sendo assim, verifico a distância e quantas pessoas conseguem usar sem afetar a qualidade da rede.
\cite{Livia}

\section{Explique cada uma das características dos dados abaixo:}
\subsection{Granularidade}

nível (detalhes/agregação)

\subsection{Consistência e Inconsistência}

Diferentes “coisas” representados pelo mesmo nome em diferentes sistemas. A mesma “coisa” representada por diferentes nomes em diferentes sistemas.

\subsection{Poluição}

Os campos podem conter espaços em branco, estar incompletos, inexatos, inconsistentes ou não identificáveis.

\subsection{Relacionamento}

É importante observar e analisar a consistência das instâncias dos objetos da estrutura problema.

\subsection{Integridade}

Deve ser observada a integridade das relações avaliadas.

\subsection{Duplicação ou Redundância}

Ocorre principalmente quando as instâncias dependem de diferentes fluxos de dados.
As variáveis Duplicadas ou Redundantes exigem maior esforço computacional e dependendo do caso podem ser reduzidas.
\cite{Livia}